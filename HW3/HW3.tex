\documentclass{article}

\usepackage{fullpage}
\usepackage[english]{babel}
\usepackage[utf8x]{inputenc}
\usepackage{amsmath}
\usepackage{amssymb}
\usepackage{graphicx}
\usepackage[colorinlistoftodos]{todonotes}
%\usepackage[]{algorithm2e}
%\usepackage[linesnumbered]{algorithm2e}
\usepackage{enumerate,url}
\usepackage{hyperref}
\hypersetup{colorlinks=true}
\usepackage{enumitem}
\usepackage[ruled]{algorithm2e} % Added by Shahrokh for Pseudocodes
\usepackage{tcolorbox}


\title{CSE 6140 / CX 4140 Assignment 3\\due Oct 16, 2020 at 11:59pm on Canvas }
\author{}
\date{}
\begin{document}
\maketitle


\section{Dominating set [12 pts]}
You’re configuring a large network of workstations, which we’ll model as
an undirected graph $G$; the nodes of $G$ represent individual workstations
and the edges represent direct communication links. The workstations all
need access to a common core database, which contains data necessary
for basic operating system functions.

You could replicate this database on each workstation; this would
make look-ups very fast from any workstation, but you’d have to manage
a huge number of copies. Alternately, you could keep a single copy of the
database on one workstation and have the remaining workstations issue
requests for data over the network $G$; but this could result in large delays
for a workstation that’s many hops away from the site of the database.

So you decide to look for the following compromise: You want to
maintain a small number of copies, but place them so that any workstation
either has a copy of the database or is connected by a direct link to a
workstation that has a copy of the database. In graph terminology, such
a set of locations is called a {\em dominating set}.

Thus we phrase the {\em Dominating Set Problem} as follows. Given the
network $G$, and a number $k$, is there a way to place $k$ copies of the database
at $k$ different nodes so that every node either has a copy of the database
or is connected by a direct link to a node that has a copy of the database?

Show that Dominating Set is NP-complete. Follow all steps we have outlined in class for a complete proof. \emph{Hint}: consider the Vertex Cover problem.

\begin{tcolorbox}
{\bf Solution:} 
\begin{itemize}
\item Step 1: Show that  {\em Dominating Set Problem} is in NP.\\
A potential solution would be $L_k = [v_1, v_2, ..., v_k]$, which is a list of $k$ vertices in the graph $G$ that was placed a copy of the database. To check if $L_k$ is a correct solution, we can loop through all the vertices in the $L_k$, store their neighbors in a hashset, and then check if the hashset has a length equal to $|V|$, the number of vertices in $G$. If we use a hashset to store $L_k$, then the worst runtime for checking a potential solution is $O(k|E|)$, where $E$ is the number of edges in $G$. Therefore, {\em Dominating Set Problem} is in NP. 

\item Step 2: Choose an NP-complete problem X.\\
{\em Set Cover: Given a set $U$ of elements, a collection $S_1, S_2, ..., S_m$ of subsets of $U$, and an integer $k$, dose there exist a collection of $\leq k$ of these sets whose union is equal to $U$? }\\ 
We know the {\em Set Cover} problem is NP-complete. 
\end{itemize}
\end{tcolorbox}

\begin{tcolorbox}
\begin{itemize}
\item Step 3: Prove that  {\em Vertex Cover}  $ \leq_{p}$ {\em Dominating Set}.
\begin{itemize}
\item Given a 
\end{itemize}
\end{itemize}
\end{tcolorbox}

\section{Frenemies [12 pts]} 

Assume you are planning a dinner party and going to invite a set of friends. However, among them, there are some pairs of persons who are enemies. You need to create a seating plan and you are wondering if it is possible to arrange this set of $n$ friends of yours around a round table such that none of the two enemies will seat next to each other. Given the set of the $n$ friends and the set of the pairs of enemies, prove that this problem is NP-Complete. Remember to follow the steps from lecture to prove NP-completeness.

You can use the fact that \texttt{Hamiltonian Cycle (HC)} is NP-complete.

\begin{tcolorbox}
{\bf Solution:}
\end{tcolorbox}

\section{Let's go hiking [26 pts]}
Alex and Baine go hiking together. They bring a bag of items and want to divide them up. For the following scenarios, decide whether the problem can be solved in polynomial time. If yes, give a polynomial-time algorithm; otherwise prove the problem is NP-complete. 
\begin{itemize}
	\item (8 pts) The bag contains $n$ items of two weights: 1lb and 2lb. Alex and Baine want to divide the items evenly so that they carry the same amount of weight. 
	
\begin{tcolorbox}
{\bf Solution:}
\end{tcolorbox}
	
	\item (9 pts) The bag contains $n$ items of different weights. Again they want to divide the items evenly. 
\begin{tcolorbox}
{\bf Solution:}
\end{tcolorbox}

	\item (9 pts) The bag contains $n$ items of different weights. They want to divide the items such that the weight difference of items they carry is less than 10lbs. 
\end{itemize}
\begin{tcolorbox}
{\bf Solution:}
\end{tcolorbox}

\textbf{Hint}: Recall Subset Sum problem: given a set $X$ of integers and a target number $t$, find a subset $Y\subset X$ such that the members of $Y$ add up to exactly $t$. 

\end{document}
